% !TeX spellcheck = en_US

%% STEP 1 : Announce your topic ONE PARAGRAPH ONLY

The labor income share is often assumed as constant by the economists. However, it has decreased during the last decades in OECD countries as emphasized by \cite{Karabarbounis2014} (see also \citealt{Elsby2013} for a detailed analysis of the US labor share).
In the same time, high-income countries such as France and the United States experiment an aging population related to the existence of a larger cohort called the \textit{baby-boomers}. Aging population and its consequences on macroeconomic variables have recently known an increasing interest as demonstrated by \cite{Sheiner2014}.
Yet, the literature has paid little attention to the impact of population dynamics on the labor share. Those two phenomena, in high-income countries, raise the question of whether the age structure of the population is a factor affecting the way in which national income is allocated between labor and capital.
Moreover, variations within the demographic structure may also affect the public policy by changing welfare state preferences. Many authors have shown the existence of intergenerational conflicts over the public budget allocation (see, for example, \citealt{Busemeyer2009}; \citealt{Sorensen2013}). The public policy determines labor market institutions which are key determinant of wage and employment and therefore the labor share.
To the best of my knowledge, this paper is the first to investigate the long-run relationship between the demographic dynamics and the labor share through the age-related conflict in public policy which determines the rules of the wage bargaining.

%% STEP 2 : Extended abstract and guideline of the paper (300 words ?)

%% Theoretical framework
I start by presenting a theoretical framework which links the age structure of the population to the labor share. I use a two-period overlapping generation (OLG) model with two types of households: young and old. Young households supply labor and earn a labor income while old households earn a capital income through their saving returns.
% Public policy
The government levies taxes on both incomes. On one hand, the government provides unemployment benefits because young people face an unemployment risk; on the other hand, the government funds health spending to the elderly. The public budget allocation is a source of intergenerational conflict because any welfare improvement for a generation is done at the cost of the other generation. The public policy is determined with a voting. The larger is a generation with respect to the other, the stronger is its political power and therefore the closer to its ideal public policy is this generation. Youth desire more redistribution and unemployment benefits.
% Wage bargaining
At the same time, the representative labor union bargains with the representative firm over wages. The out-of-work options of young agents are positively affected by the level of unemployment benefits but negatively by the tax rate. These options enable the representative union to bargain greater wages. However, greater wages reduce the labor demand of the representative firm and thus increases the capital-per-worker and the production-per-worker. The labor share can be defined as the ratio between the wage rate and the production-per-worker. Consequently, the effect of greater wages on the labor share depends on the value of the capital-labor elasticity of substitution.
% Link
At the equilibrium, the state of the economy is determined by the interaction of the voting and the wage bargaining. The total effect of demographic dynamics on the labor share passes through three variables: the capital stock determined by the savings of the previous young generation; the labor supply which is taken into account in the wage bargaining; and the youth political power that define the level of out-of-work options and thus the ability of workers to increase their wages. Despite the fact that the key variables of the model are identified, the total effect is ambiguous and depends on parameter values.

%% Quantitative analysis
% Mechanisms
To deal with the indeterminacy of the qualitative effect, I realize a quantitative analysis for France and the United-States. The calibration of the parameters leads to an elasticity of substitution between capital and labor greater than one. Both input factors are gross substitute. The model is able to replicate the data over the last decades and predicts a slight rise of the labor share due to the aging of the baby-boomers' cohort. This cohort drives the public policy agenda and so the economy. When the baby-boomers are young, their massive entrance on the labor market increases the labor supply. They also shape the institutions in their favor through their votes which permit the union to bargain greater wages. As a response, the firm shifts away from labor toward capital to thwart the workers' ability to grab part of the income. Thus, the labor share declines. Once this generation becomes old, the mechanism is reversed. However, the expected resurgence of the labor share is offset by the capital accumulation fostered by the extensive savings of the baby-boomers when they were young. In the very long run, the labor share decreases with the aging of the population.

% Counterfactual, decomposition and dicussion
I also make counterfactual model predictions in order to quantify the respective role of each determinants and channels. The determinants of the demographic dynamics' impact on the labor share are the population growth and the survival rate, i.e. the life expectancy. The rising survival rate is the dominant explanatory factor in both countries. Then, the age structure of the population affects the labor share in two different ways: directly through the labor supply and the capital accumulation, and indirectly through the public policy. Model predictions suggest that the indirect channel should play a considerable role in the next decades due to the retirement of the baby-boomers.

% Discussion
Finally, I discuss two points of this paper. Firstly, even though baby-boomers appeared as income losers over the last decades because the labor share has fallen. They were actually the winners once net income is considered due to the implementation of a redistributive public policy. Once they retire, they are still the winners of public budget allocation's intergenerational conflict because their children fund their health spending. Secondly, I find that an increase of the retirement age in the next decades should lead to a decline of the labor share due to capital over-accumulation, but an increase in the very long-run.

%%% OLD VERSION
%% STEP 3 : What the paper does & how it fills the gap or solves the problem that you have identified
%% What the paper does
%In the continuity of their approach, I investigate the negative correlation between the aging population and the labor share. I consider an OLG model in a framework with probabilistic voting and wage bargaining. In addition of a change in capital stock, the labor share is also affected by outside options in the wage bargaining. These outside options are determined through the public policy where it exists an age-related conflict between young and old. 
%I realize a quantitative analysis for France and the United-States. I also perform an aging-effect decomposition to analyze to which extent demographic determinants affect the labor share and through which channels. As \cite{Gonzalez-Eiras2012}, there are two determinants of demographic dynamics: population growth and survival rate (i.e. life expectancy). But also two transmission mechanisms: a direct one that is reflected in capital accumulation and labor supply ; and an indirect one through the public policy instruments.
%
%% Preview of the results
%Quantitative results suggest that the model is able to replicate long-run dynamics of the labor share in France and United-States. Demographic dynamics explain relatively better the French data pattern with respect to US data. This is due to the larger size of the welfare state and the greater disparity of cohort sizes in France. 
%The declining labor share of the past decades was mainly driven by the baby-boomers cohort. Model predictions suggest that the aging of the baby-boomers cohort should generate a modest increase of the labor share for next decades.
%I find that the rising survival rate is a much more explaining factor of labor share dynamics than the declining population growth. I also find that the direct effect dominates the indirect one until 2010 and is dominated thereafter.
%Although the labor income share was declining due to policies implemented by baby-boomers, they were not the losers over this period because they extracted part of the income of the elderly through redistribution.

%% STEP 3 : Review previous research on your topic RELATED TO WHAT YOU HAVE DONE ! To which part of the literature you add something ?

%% Labor share
This paper is related to the extensive literature on the labor share (see, for example, \citealt{Blanchard1997}; \citealt{Caballero1998}; \citealt{Acemoglu2003}; \citealt{Karabarbounis2014}; \citealt{Autor2019}). Multiple determinants have been analyzed by economists to explain its decline over the last decades. Among them, the role of institutions and the so called biased technical change are leading.\footnote{Although the globalization has also recently received sizable research interest as a determinant of the labor share (see, for example, \citealt{Jayadev2007}; \citealt{Pica2010}; \citealt{Young2018}; and \citealt{Autor2019}).}
The institutional context argument started with \cite{Blanchard1997} to explain the persistence of shocks on the labor market. In principle, adverse supply shocks of the 1970s would have had an impact on employment and the labor share but only in the short run. Due to labor market institutions, such as adjustment costs, these shocks generate long lags in labor demand and thereby their persistence in the long-run. This argument relies on the previous work of \cite{Bentolila1990} who show the concern of firing cost in the firms' ability to hire and fire. Such a labor market institution constrains the firms and leads to increase the labor share in the short run. In addition to adjustment costs, \cite{Bentolila2003} also highlight the role of workers' bargaining power to explain the gap between the marginal product of labor and the wage.
These pro-labor income institutions are a burden to firms because they limit the firms' ability to optimize input factors' allocation but also because they enable the workers to obtain a high income share in the short-run. \cite{Caballero1998} incorporate the long-run response of firms in order to thwart workers empowerment. This response is the substitution of labor with capital through biased technical change.\footnote{\cite{Acemoglu2002} shows that factor abundance and factor prices are key determinant of the direction toward which factor the technical change is biased. In line with this result, \cite{Karabarbounis2014} show that the decrease in the relative price of investment goods induced firms to shift away from labor and toward capital. The price of investment goods is relative to the one of consumption goods in their paper. However, it can also be interpreted as the relative price of capital with respect to labor since wages are correlated with the consumer price level.} Others have developed models with factor-saving innovation (see, for example, \citealt{Zuleta2008}; \citealt{Peretto2013}).




%% Aging on macro variables

%% Politico-Economic equilibrium

%% Capital-labor elasticity of substitution

%% Retirement age ?


\newpage



% Caballero and Hammour (1998)
\cite{Caballero1998} incorporate the long-run response of firms to labor market institutions shaped in favor of workers. This response is the substitution of labor with capital. However, they do not look upstream to determine the reasons of such a labor market institutional context. In high-income countries, such as France and the United-States, the after-war period has been characterized by a colossal increase of the working-age population. Either with the so called baby-boomers cohort or the immigration. During the 1970s and the 1980s, this generation may have shaped the labor market institutions in their favor because they had enough political weight to do so. This period has been characterized by rising unionization, employment protection and unemployment benefits% (\textbf{graph?})
. All of these labor market institutions has raised real wages and therefore the labor cost for firms. In such a context, the argument of \cite{Caballero1998} fits perfectly. Firms have shifted from labor to capital in order to thwart workers empowerment.




%% Labor share literature
This paper is related to the extensive literature on the labor share. Multiple determinants have been analyzed by economists to explain its decline. The three main explaining factors are institutions, biased technical change and globalization. Furthermore, these factors are closely linked.
% Institutions
Institutions affect the labor share through the labor market structure. \cite{Blanchard1997} points out the role played by institutions in the persistence of shocks. He references and examines the adverse shifts in labor supply during the 1970s and in labor demand during the 1980s that occurred in Continental European countries. In principle, adverse supply shocks of the 1970s would have had an impact on employment and the labor share but only in the short run. Due to labor market institutions, such as adjustment costs, these shocks generate long lags in labor demand and thereby their persistence in the long-run. \cite{Bentolila1990} show the concern of firing cost in the firms' ability to hire and fire. Such a labor market institution constrains the firms and leads to increase the labor share in the short run. In addition to adjustment costs, \cite{Bentolila2003} also highlight the role of workers' bargaining power to explain the gap between the marginal product of labor and the wage.
% To summarize, these authors have shown that labor market institutions increase the labor share, at least in the short run.

% Biased technical change
Biased technical change is also a determinant of the labor share. \cite{Acemoglu2002} argues that endogenous technical change can be directed towards one factor or an other, thus generating a bias. The targeted factor is determined by two competing effects: the price effect and the market size effect. The former favors the scarce factor while the latter promotes the abundant factor. The relative strength of those two effects are closely linked to the value of the elasticity of substitution between factors. Therefore, factor abundance and factor prices are determinant of the labor share. In line with this result, \cite{Karabarbounis2014} show that the decrease in the relative price of investment goods induced firms to shift away from labor and toward capital. Others have developed models with factor-saving innovation (see, for example, \citealt{Zuleta2008}; \citealt{Peretto2013}).

% Caballero and Hammour (1998) : link institutions and BTC
As previously mentioned, the determinants of the labor share are closely linked. \cite{Caballero1998} show the impact of labor market institutions on the capital accumulation. They argue that biased technical change is a response of firms in the long run to the ability of workers to obtain a high income share in the short run. This ability is owing to pro-labor income institutions. These institutions allow workers to grab part of the capital income in the short run. In order to thwart this appropriation, firms substitute labor with capital in the long run.

% Globalization
Some authors have shown that globalization has a negative effect on the labor share through different aspects. \cite{Elsby2013} mention offshoring as one of them. Due to trade liberalization, labor intensive components of the US supply chain are outsourced abroad where labor factor is cheaper. \cite{Krugman2008} and \cite{Bassanini2014} also support this argument.\footnote{However, \cite{Krugman2008} argues that offshoring has a modest impact because developing countries are specialized in niche industries. But also due to vertical specialization industries, meaning that only simplest tasks are performed abroad and thereafter final products are assembled in developed countries.} An other explanation belongs to trade and capital account openness. The increasing mobility of goods and capital has negative consequences on the labor share. By either decreasing directly the wage bargaining power (see \citealt{Jayadev2007} and \citealt{Young2018}) or giving incentive to implement labor market reforms due to capital competition between countries as emphasized by \cite{Pica2010}. These reforms then reduces wage bargaining power. Finally, a more recent reasoning is developed by \cite{Autor2019} about ``superstar firms''. Due to globalization, the most productive firms dominates their market. These superstar firms are characterized by high markups and low labor share. They reduce the aggregated labor share because they represent a growing share of the output.









%% Demographic dynamics literature
This paper is also related to the literature on demographic dynamics and their implications on economic variables. This topic is not new in itself, many authors have already looked at it at micro and macro-levels (see \citealt{Clark1978} for a survey). The renewal is mainly motivated by the current and expected shift of the population structure in high-income countries.\footnote{See \cite{Bloom2016} for a recent survey of the determinants of population aging.} This phenomenon presents serious challenges in many fields of the economy. Some authors have examined its impact on economic growth (see \citealt{VanGroezen2005}; \citealt{Soares2005}; \citealt{Bloom2010}; \citealt{Lee2010}). Others have investigated the sustainability of pension systems in such a context (see \citealt{Ono2003}; \citealt{DelaCroix2013}; \citealt{Philipov2014}) and discussed about the optimality and feasibility of pension reforms (see \citealt{Pecchenino1997}; \citealt{Sinn2003}). Related to the pension issue, the legal age of retirement is also probed (see \citealt{Futagami2001}; \citealt{Dedry2017}).

% Sorensen, Busemeyer and Jager
Besides all these challenges raised by population aging, the public budget allocation remains the main underlying political issue. This issue is related to the presence of an age-related conflict within the public policy. The youth and their elderly do not benefit from the same public policy instruments neither have the same type of income. Therefore, preferences are likely to change over the life-cycle. On one hand, \cite{Sorensen2013} shows that elderly people desire less spending in education while they are in support of more in health and pension.\footnote{\cite{Sorensen2013} uses cross-sectional survey data for 22 countries from the International Social Survey Program (ISSP). There are four Role of Government surveys corresponding to the years 1985, 1990, 1996 and 2006.} However, the author claims that these life-cycle effects are quite small. On the other hand, \cite{Busemeyer2009} find sizable age-related differences in public policy preferences.\footnote{\cite{Busemeyer2009} use cross-sectional survey data for 14 OECD countries from the 1996 ISSP Role of Government dataset.} Although both studies disagree on the magnitude of the conflict, they both claim that such a conflict exists. They also agree about the heterogeneity among countries of these life-cycle effects. \cite{Jager2016} also assess the impact of demographic aging on public investment.\footnote{\cite{Jager2016} use panel data of 19 OECD countries between 1971 and 2007.} They find a negative long-run relationship between both relying on the fact that older people discount more future payoffs with respect to young people.\footnote{See, for example, \cite{Harrison2002}; \cite{Read2004}. See also \cite{Huffman2017} for a more recent empirical study with a decomposition by characteristics of time discounting in old population.}

%% Politico-economic models
This paper relates to the literature on politico-economic models. This literature has started following the work of \cite{Lindbeck1987} on probabilistic voting. These models are useful to examine the relationship between redistribution policies and growth (see, for example, \citealt{Alesina1994}; \citealt{Persson1994}; \citealt{Krusell1997}). However, they are rather focused on intra-generational government budget allocation conflicts than inter-generational. Lately, some authors have brought these models back into use in order to analyze the age-related conflict between generations.
% Lancia & Prarolo (2012)
\cite{Lancia2012} develop an overlapping generations (OLG) model with three types of agents: young, adult and old. They focus their analysis on economic growth and find that an increasing life expectancy is a growth-enhancing factor. Economic growth is determined by two components: human capital growth and total factor productivity growth. First, young agents inherit the average human capital level of their parents and decide to invest on their education level. The higher is the expected life expectancy, the greater is the incentive to educate. Investing in education increases human capital. Second, all agents vote to determine the innovation public policy that fosters total factor productivity. This investment is done at the cost of public pensions. Thus, elderly are necessary opposed to innovation. When both, young and adults, are net winners from public investment policy, they form a coalition in the political process to adopt such a policy. Therefore, aging has two opposite effects on growth. On one hand, it generates incentives to human capital accumulation and innovation. On the other hand, it increases the political weight of elderly who are against innovation and makes policies more difficult to implement.
% Gonzalez-Eiras and Niepel (2012)
With the same type of conflict on transfers to retired old households and public investment, \cite{Gonzalez-Eiras2012} analyze implications for per-capita growth. However, they also endogenize the retirement age in a politico-economic equilibrium. They distinguish two effects: a direct one through the savings rate, labor supply and capital accumulation ; and an indirect one through the age-related conflict within the public policy to determine the levels of taxes, government spending and retirement age. They predict an increase of tax rates and the retirement age in OECD countries to offset population aging and thereby boost per-capita growth.

%% STEP 3 : Identify the gaps or the problems with previous research

Despite the existing literature on the relationship between demographic dynamics and growth, the impact of demography on the income allocation between capital and labor has been understudied. 
% Schmidt and Vosen (2013)
To the best of my knowledge, \cite{Schmidt2013} is the only one paper to examine the impact of the demography on the labor share. Using an OLG model with pension system, they show that an aging population leads to more saving and so more capital. In a closed economy, the effect on the labor share depends on the value of the capital-labor elasticity of substitution. For an elasticity greater (resp. smaller) than unity, the effect is negative (resp. positive). While in a small economy with perfect capital mobility, they argue that households invest abroad because domestic interest rates fall. As a result, the increasing net foreign assets income shrinks the labor share. Their paper points out a link between the population aging and the labor share through capital accumulation. However, they do not consider any age-related conflict within the public budget allocation.

% Caballero and Hammour (1998)
\cite{Caballero1998} incorporate the long-run response of firms to labor market institutions shaped in favor of workers. This response is the substitution of labor with capital. However, they do not look upstream to determine the reasons of such a labor market institutional context. In high-income countries, such as France and the United-States, the after-war period has been characterized by a colossal increase of the working-age population. Either with the so called baby-boomers cohort or the immigration. During the 1970s and the 1980s, this generation may have shaped the labor market institutions in their favor because they had enough political weight to do so. This period has been characterized by rising unionization, employment protection and unemployment benefits% (\textbf{graph?})
. All of these labor market institutions has raised real wages and therefore the labor cost for firms. In such a context, the argument of \cite{Caballero1998} fits perfectly. Firms have shifted from labor to capital in order to thwart workers empowerment.


%% FINAL STEP : Describe the organization of your paper

The paper is organized as follows. Section \ref{sec:model} presents the theoretical framework. Section \ref{sec:quantitative_analysis} presents the quantitative analysis. It starts with the analysis of model predictions and mechanisms. I then perform an aging-effect decomposition with counterfactual simulations. Section \ref{sec:discussion} discusses some results of the paper. Section \ref{sec:conclusion} concludes.

