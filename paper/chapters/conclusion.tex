The literature on the labor share has emphasized the role of many factors such as the biased technical change or the labor market institutions. I built an OLG model in which labor market institutions are endogenously determined through public policy and affect the wage bargaining. Using this model, I analyzed the impact of demographic dynamics on labor share's long-term dynamics in France and the United-States. Numerical simulations are able to replicate data for both countries since the 1970s.
% I use the argument developed by \cite{Caballero1998} where firms shifted away from labor as a response of worker's ability to obtain a relatively high labor share. Workers and unions are able to do so due to labor market institutions.

%% Mechanisms

%I highlight the different determinants and channels through which demographic dynamics affect the labor share.
%The model is able to replicate data for France and the United-States. 
Model predictions suggest that the decline of the labor share during the last decades was driven by cohort-size effects. 
For France, the baby-boomers cohort seems to drive the public policy and thus the way in which national income is allocated between labor and capital.
When the baby-boomers cohort enters on the labor market, they shape the labor market institutions in their favor because they face an unemployment risk. Opportunistic political candidates implements public policy desired by this cohort. 
Thus, these more protective labor market institutions rise workers' outside option which enables unions to bargain greater wages. 
As a response, firms shift away from labor and use more capital. This mechanism is due to the fact that both input factors are gross substitute and that the firms have the prerogative to hire and fire.
The unemployment rate rises and so does the production-per-worker. 
The increase of the production-per-worker offsets the one of the wage rate. 
Therefore, the labor share declines over the end of the twentieth century. 
Thereafter, the baby-boomers retire and trigger the opposite mechanism. 
Although the public policy shrinks the outside option of workers within the wage bargaining which should reduce wages and raise employment. 
The capital stock has sharply increased due to massive savings of the baby-boomers when they were young. 
The important available capital stock partially thwart the reverse mechanism and the labor share does not recover its past level.
The model predicts a slight resurgence of the labor share for the next decades in France and a stagnation in the United-States.

%% Results

Model predictions suggest that the survival rate dynamics have a larger impact on the labor share than the population growth dynamics. I also decompose this impact between two channels: the direct cohort-size effect and the indirect cohort-size effect. The former dominates for France when the baby boomers are young, while the latter takes the lead once this cohort retires. Although the labor share declined due to public policies implemented by baby boomers, their after-tax income did not. Because they also increased the redistribution through taxes. An increase of the retirement age should decrease the labor share in the following decades due to the over-accumulation of capital and the raising political power of the youth. However, the labor share is expected to be greater in the very long-run.

%% Main assumptions
The two main hypothesis of the paper are about the elasticity of substitution between capital and labor and the right-to-manage specification for the wage bargaining. Recent estimates suggest that this elasticity may be greater than one (see \citealt{Karabarbounis2014}) but an other part of the literature has found it below one, particularly for the United-States (see, for example, \citealt{Antras2004}; \citealt{Chirinko2008}). Moreover, I do not include any form of biased technical change within the model. This is voluntary in order to develop an other theory on the labor share's decline based on demographic dynamics. It could be the case that biased technical change is also driven by demographic dynamics through the \textit{grability} of workers to seize part of the rent. This grability may be generated by some cohorts which are sufficiently numerous to shape labor market institutions in their favor and therefore in favor of labor. I let the investigation of a potential endogenous biased technical change induced by demographic dynamics for further research.





