% !TeX spellcheck = en_US

%% STEP 1 : Announce your topic ONE PARAGRAPH ONLY

The labor income share is often assumed as constant by the economists. However, it has decreased during the last decades in OECD countries as emphasized by \cite{Karabarbounis2014}.
% (see also \citealt{Elsby2013} for a detailed analysis of the US labor share).
\cite{Caballero1998} show that firms substitute labor with capital through biased technical change to thwart workers empowerment.
At the same time, high-income countries such as France and the United States have experienced a population aging related to the existence of the larger cohort of the \textit{baby-boomers}.
%Aging population and its consequences on macroeconomic variables have recently received an increasing interest; see \cite{Sheiner2014}.
Yet, the literature has paid little attention to the impact of population dynamics on the labor share. 
%The aging of the population and the declining labor share, in high-income countries, raise the question of whether the age structure of the population is a factor affecting the way in which national income is allocated between labor and capital.
Moreover, variations of the demographic structure may also affect public policy by changing welfare state preferences. Many authors have shown the existence of inter-generational conflicts over public budget allocation (see, for example, \citealt{Busemeyer2009}; \citealt{Sorensen2013}). 
Public policy determines labor market institutions which are key determinants of wage and employment and therefore the labor share. Thus, the shift away from labor toward capital may be due to labor market institutions endogenously determined by the age structure of the population.
To the best of my knowledge, this paper is the first to investigate the long-run relationship between the demographic dynamics and the labor share through the inter-generational conflict.

%% STEP 2 : Extended abstract and guideline of the paper (300 words ?)

%% Theoretical framework
I start by presenting a theoretical framework which links the age structure of the population to the labor share. I use a two-period overlapping generations (OLG) model with two types of individuals: young and old. Young individuals supply labor and earn a labor income while old households earn capital income, the return of their savings.
% Public policy
The government levies taxes on both incomes and uses them to provide unemployment benefits and fund health spending on the elderly. The public budget allocation is a source of age-related conflict because any welfare improvement for a generation is done at the expense of the other generation. Public policy is endogenous and determined through voting. The larger is a generation with respect to the other, the stronger is its political power and therefore the closer to its desired public policy is this generation. Youth desire more redistribution and unemployment benefits.
% Wage bargaining
Meanwhile, the representative labor union bargains with the representative firm over wages. The out-of-work options of young agents are positively affected by the level of unemployment benefits but negatively by the tax rate. These options enable the representative union to bargain greater wages. However, greater wages reduce the labor demand of the representative firm and thus increases the capital-per-worker and output-per-worker. The labor share can be defined as the ratio between the wage rate and output-per-worker. Consequently, the effect of greater wages on the labor share depends on the value of the capital-labor elasticity of substitution.
% Link
The equilibrium is determined by the interaction of the voting and the wage bargaining. The total effect of demographic dynamics on the labor share passes through three variables: the capital stock determined by the savings of the previous young generation; the labor supply which is taken into account in the wage bargaining; and the youth political power that defines the level of the out-of-work option and thus the ability of workers to increase their wages. However, the total effect is ambiguous and depends on parameter values.

%% Quantitative analysis
% Mechanisms
To deal with the ambiguity of the qualitative effect, I provide a quantitative analysis for France and the United-States. The calibration of the parameters leads to an elasticity of substitution between capital and labor greater than one. Thus, both input factors are gross substitutes, meaning that firms are able to substitute labor with capital for a given level of output. The model is able to replicate the data over the last decades and predicts a slight rise of the labor share due to the aging of the baby-boomers' cohort. This cohort drives the public policy agenda and hence the labor share. When the baby-boomers are young, their massive entrance in the labor market increases the labor supply. They also shape institutions in their favor through the voting process, leading to greater wages. As a response, firms shift away from labor toward capital to thwart the workers' ability to grab part of the income. Thus, the labor share declines. Once this generation becomes old, the mechanism is reversed. However, the expected increase in the labor share is dampened by the capital accumulation fostered by the extensive savings of the baby-boomers. 
% In the very long run, the labor share decreases with the aging of the population.

% Counterfactual, decomposition and dicussion
I also quantify the role of each determinant and channel. 
Demographic dynamics are determined by the population growth and the survival rate, i.e. the life expectancy. The rising survival rate is the dominant explanatory factor in both countries. Then, the age structure of the population affects the labor share in two different ways: directly through the labor supply and the capital accumulation, and indirectly through the endogenous public policy. Model predictions suggest that the indirect channel should play a considerable role in the next decades due to the retirement of the baby-boomers.

% Discussion
Lastly, two aspects are worth noting. First, even though baby-boomers appear as income losers over the last decades because the labor share has fallen, they were actually the winners once net income is considered due to the implementation of a redistributive public policy. Once they retire, they are still the winners of public budget allocation's inter-generational conflict because the mechanism works the other way around.
Second, I find that an increase of the retirement age in the next decades should lead to a decline of the labor share due to capital over-accumulation.
Young agents expected to remain longer retired and therefore saved more, leading to an over-accumulation of capital. At the same time, the youth's political power increases and so does the wage. Thus, the firm substitute labor with capital. However, once the capital comes back to its optimal accumulation path, so in the very long-run, the labor share increases.

%% STEP 3 : Review previous research on your topic RELATED TO WHAT YOU HAVE DONE ! To which part of the literature you add something ?

%% Labor share
This paper is related to the extensive literature on the labor share (see, for example, \citealt{Blanchard1997}; \citealt{Caballero1998}; \citealt{Acemoglu2003}; \citealt{Karabarbounis2014}; \citealt{Autor2019}). Multiple determinants have been analyzed by economists to explain its decline over the past decades, notably the role of institutions and the biased technical change.\footnote{Although the globalization has also recently received sizable research interest as a determinant of the labor share (see, for example, \citealt{Jayadev2007}; \citealt{Pica2010}; \citealt{Young2018}; and \citealt{Autor2019}).}
The institutional context argument was first put forward by \cite{Blanchard1997} to explain the persistence of shocks to the labor market. In principle, the adverse supply shocks of the 1970s should have had an impact on employment and the labor share only in the short run. Due to labor market institutions, such as adjustment costs, these shocks generated long lags in labor demand and thereby their persistence in the long-run.
%This argument relies on the previous work of \cite{Bentolila1990} who show the concern of firing cost in the firms' ability to hire and fire. Such a labor market institution constrains the firms and leads to increase the labor share in the short run. 
In addition to adjustment costs, \cite{Bentolila2003} also highlight the role of workers' bargaining power to explain the gap between the marginal product of labor and the wage.
These pro-labor income institutions are a burden to firms because they limit the firms' ability to optimize input factors' allocation but also because they enable the workers to obtain a high income share in the short-run. \cite{Caballero1998} incorporate the long-run response of firms in order to thwart workers empowerment. This response is the substitution of labor with capital through biased technical change.\footnote{\cite{Acemoglu2002} shows that factor abundance and factor prices are key determinants of the direction toward which factor the technical change is biased. In line with this result, \cite{Karabarbounis2014} show that the decrease in the relative price of investment goods induced firms to shift away from labor and toward capital.
%The price of investment goods is relative to the one of consumption goods in their paper. However, it can also be interpreted as the relative price of capital with respect to labor since wages are correlated with the consumer price level.
} Others have developed models with factor-saving innovation (see, for example, \citealt{Zuleta2008}; \citealt{Peretto2013}).
I do not incorporate any biased technical change in the model in order to analyze the role of the demographic dynamics. I show that the model is able to replicate labor share's dynamics without biased technical change. Therefore, it suggests that biased technical change may be the firms' response to the workers empowerment fostered by changes in the age structure of the population. I use the argument developed by \cite{Caballero1998} by looking at an upstream step to determine the reasons of such a labor market institutional context. I claim that labor market institutions are the result of the aggregation of public policy preferences. Therefore, the age structure of the population does matter to explain the dynamics of the labor share.

%% Aging on macro variables
An interest in the impact of demographic dynamics on economic variables is not new. Many authors have looked at it at the micro and macro-levels.\footnote{see \cite{Clark1978} for a survey; and \cite{Bloom2016} for a recent survey of the determinants of population aging.} 
This phenomenon presents serious challenges in many fields of the economy. Some authors have examined its impact on economic growth (see \citealt{VanGroezen2005}; \citealt{Soares2005}; \citealt{Bloom2010}; \citealt{Lee2010}). Others have investigated the sustainability of pension systems in such a context (see \citealt{Ono2003}; \citealt{DelaCroix2013}; \citealt{Philipov2014}) and discussed the optimality and feasibility of pension reforms (see \citealt{Pecchenino1997}; \citealt{Sinn2003}). Related to the pension issue, the legal age of retirement is also probed (see \citealt{Futagami2001}; \citealt{Dedry2017}). Despite the existing literature on population aging, the impact on the income allocation between capital and labor has been understudied. To the best of my knowledge, \cite{Schmidt2013} is the only existing paper looking at the impact of population aging on the labor share. They use an OLG model with a pension system to show that an aging population leads to more saving and hence more capital. For a capital-labor elasticity of substitution greater (resp. smaller) than unity, the effect on the labor share is negative (resp. positive). Their paper points out a link between the population aging and the labor share through capital accumulation.\footnote{They also show that population aging reduces the labor share in a small economy with perfect capital mobility, regardless of the value of capital-labor elasticity of substitution. Households invest abroad because domestic interest rates fall. As a result, the increasing net foreign assets income shrinks the labor share.} My paper includes this mechanism and reaches the same conclusions on it. However, I add a new mechanism through the inter-generational conflict over public budget allocation and show that it accounts for more than the half of the dynamics.

% Public policy preferences
This paper relies on recent empirical studies showing that public policy preferences are likely to change over the life-cycle. Because the youth and the elderly do not benefit from the same public policy instruments nor have the same income sources (see, for example, \citealt{Busemeyer2009}; \citealt{Sorensen2013}).\footnote{\cite{Sorensen2013} uses cross-sectional survey data for 22 countries from four Role of Government surveys (1985, 1990, 1996 and 200) of the International Social Survey Program (ISSP). He shows that elderly people desire less spending in education while they are in support of more in health and pension. But he claims that these life-cycle effects are quite small. \cite{Busemeyer2009} use cross-sectional survey data for 14 OECD countries from the 1996 ISSP Role of Government dataset. On the contrary, they find sizable age-related differences in public policy preferences.} Although these studies disagree on the magnitude of the conflict, they all claim that such a conflict does exist. They also agree on the magnitude heterogeneity across countries. \cite{Jager2016} find a negative long-run relationship between population aging and public investment because older individuals discount more future payoffs with respect to young individuals.\footnote{\cite{Jager2016} use panel data of 19 OECD countries between 1971 and 2007. See also, \cite{Harrison2002}; \cite{Read2004}, for estimations of discount rates. It is also worth mentioning the recent empirical study of \cite{Huffman2017} in which they decompose the characteristics of time discounting in old population.} 
I make use of recent findings on life-cycle public policy preferences to generate an inter-generational conflict over the public budget allocation. For simplicity, I summarize it on two dimensions: the unemployment benefits and the health spending. Although the unemployment benefits are necessary to generate an interaction with the labor market through the wage bargaining. Health spending could be replaced by any type of government spending that benefits the elderly.

%% Politico-Economic equilibrium
This paper also relates to the literature on politico-economic models. This literature follows the work of \cite{Lindbeck1987} on probabilistic voting. These models are useful to examine the relationship between redistribution policies and growth.\footnote{See, for example, \cite{Alesina1994}; \cite{Persson1994}; \cite{Krusell1997}.} The first models were rather focused on intra-generational government budget allocation conflicts, while recent papers such as \cite{Lancia2012} are concerned with inter-generational conflicts. \cite{Lancia2012} develop a politico-economic equilibrium model in which aging has two opposite effects on growth. On one hand, it generates incentives for human capital accumulation and innovation, on the other, it increases the political weight of the elderly who are against innovation and makes policies more difficult to implement.\footnote{Their OLG model has three types of agents: young, adult and old. The economic growth is determined by two components: human capital growth and total factor productivity growth. First, young agents inherit the average human capital level of their parents and decide to invest on their education level. The higher is the expected life expectancy, the greater is the incentive to educate. Investing in education increases human capital. Second, all agents vote to determine the innovation public policy that fosters total factor productivity. This investment is done at the cost of public pensions. Thus, elderly are necessary opposed to innovation. When both, young and adults, are net winners from public investment policy, they form a coalition in the political process to adopt such a policy.} 
With the same type of conflict on transfers to retired old households and public investment, \cite{Gonzalez-Eiras2012} analyze implications for per-capita growth.\footnote{However, they also endogenize the retirement age in a politico-economic equilibrium.} My approach is closely linked to theirs but applied to the labor share. They distinguish two effects: a direct one through the savings rate, labor supply and capital accumulation, and an indirect one through the age-related conflict to determine taxes, government spending and the retirement age. They predict an increase in the tax rate and the retirement age in OECD countries to offset population aging which should boost per-capita growth. I use the same decomposition into the direct and indirect channels. I also decompose the aging of population in the decline in population growth and the increase in the survival rate, i.e. the life-expectancy.
I show that the age structure of the population affects factor shares. Yet, most of the papers that look at the impact of population aging on the economic growth use a Cobb-Douglas production function where factor shares are constant. Therefore, I suggest that some conclusions should be reconsider inline with this mechanism.

%% Capital-labor elasticity of substitution ?

%% Limits ?

The two main hypothesis of the paper are about the elasticity of substitution between capital and labor and the right-to-manage specification for the wage bargaining. Recent estimates suggest that this elasticity may be greater than one (see \citealt{Karabarbounis2014})\footnote{\cite{Caballero1998} also use a relatively high value of the capital-labor elasticity of substitution, about 6.00, to simulate French data.} but an other part of the literature has found it below one, particularly for the United-States (see, for example, \citealt{Antras2004}; \citealt{Chirinko2008}). 
Moreover, I do not include any form of biased technical change within the model. This is voluntary in order to develop an other theory on the labor share's decline based on demographic dynamics. It could be the case that biased technical change is also driven by demographic dynamics through the \textit{grability} of workers to seize part of the rent. This grability may be generated by some cohorts which are sufficiently numerous to shape labor market institutions in their favor and therefore in favor of labor.

%% FINAL STEP : Describe the organization of your paper

The paper is organized as follows. Section \ref{sec:model} presents the theoretical framework. Section \ref{sec:quantitative_analysis} provides the quantitative analysis. It starts with the analysis of the model's predictions and mechanisms. I then perform an aging-effect decomposition with counterfactual simulations. Section \ref{sec:discussion} discusses some results of the paper. Section \ref{sec:conclusion} concludes.